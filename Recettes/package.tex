\documentclass{article}
\usepackage[fontsize=12]{scrextend}
\usepackage[T1]{fontenc}
\usepackage{lmodern}
\usepackage[french]{babel}

%tmp
\usepackage{blindtext} 
\usepackage{multicol} %plusieurs colonnes
\usepackage{eurosym} %euro
\usepackage{framed}
\usepackage{titletoc} 
\usepackage[export]{adjustbox} %min width
\usepackage{easyfig} %figures

%commandes perso
\usepackage{mathtools} %pour paireddelimiter
\usepackage{stmaryrd}  %pour \llbracket
\DeclarePairedDelimiter{\ceil}{\lceil}{\rceil}
\DeclarePairedDelimiter{\floor}{\lfloor}{\rfloor}
\DeclarePairedDelimiter{\abs}{\lvert}{\rvert}
\DeclarePairedDelimiter{\ii}{\llbracket}{\rrbracket}
\DeclarePairedDelimiter{\ie}{\llbracket}{\llbracket}
\DeclarePairedDelimiter{\ei}{\rrbracket}{\rrbracket}
\DeclarePairedDelimiter{\ee}{\rrbracket}{\llbracket}

%utilitaire
\usepackage{embedfile} %joindre les fichiers source au pdf
\usepackage{datetime}  %la date
\usepackage{xargs}     %fonction avec plusieurs paramètres facultatifs 
\usepackage{lipsum}    %générateur de text
%\usepackage{tgbonum}   %font

%pied de page 
\usepackage{fancyhdr}  %pied de page
\setlength{\headheight}{13pt}
\usepackage{lastpage}  %numéro de la dernière page

%Packages Mise en page
\usepackage{babel}     %footnote et traductions
\usepackage{tabto}     %faire des tabulation
\usepackage{titling}   %gérer la position du titre
\usepackage{setspace}  %double espaces
\usepackage{changepage}%régler les blocs des sections
\usepackage[pdfencoding=auto, psdextra, hidelinks]{hyperref}%hyperlink
\usepackage{cleveref}  %reférence pour lemme et thm
\makeatletter          %éviter les erreurs pour les liens hyperef et le mode math
\pdfstringdefDisableCommands{\let\HyPsd@CatcodeWarning\@gobble}
\makeatother

%surligner et couleurs
%\usepackage{colortbl}  %dans un tableau
\usepackage{color}     %couleurs
\usepackage{soul}      %surligner
%\usepackage[table]{xcolor}%couleurs
\usepackage{xcolor}%couleurs
\usepackage{soulutf8}  %surligner avec utf8

%Packages mathématiques
\usepackage{amsfonts}  %police d'écriture
\usepackage{amsmath}   %Equations
\usepackage{amsthm}    %théoreme
\usepackage{amssymb}   %symboles
\usepackage{mathrsfs}  %police
\usepackage{pifont}    %police d'écriture
\usepackage{subcaption}%sous-légende
\usepackage{textcomp}  %écrire les +_ (\textpm)
\usepackage{txfonts}   %coeur

%Théorème lemme preuves etc
\theoremstyle{definition}
\newtheorem{theorem}{Theorem}
\newtheorem{corollaire}[theorem]{Corollaire}
\newtheorem{definition}[theorem]{Définition}
\newtheorem{exemple}[theorem]{Exemple}
\newtheorem{hypo}[theorem]{Hypothèse}
\newtheorem{lemme}[theorem]{Lemme}
\newtheorem{methode}[theorem]{Méthode}
\newtheorem{notation}[theorem]{Notation}
\newtheorem{proposition}[theorem]{Proposition}
\newtheorem{propriete}[theorem]{Propriété}
\newtheorem{prop}{Proposition}
\newtheorem{remarque}[theorem]{Remarque}
\newtheorem{reponse}[theorem]{Réponse}
\newtheorem{theoreme}[theorem]{Théorème}
\theoremstyle{definition}

%Packages Tableaux
\usepackage{array}    %Tableaux maths
\usepackage{arydshln} %Lignes en pointillés
\usepackage{fancybox} %Boites
\usepackage{fancybox} %encadrage
\usepackage{multicol} %Colonnes
\usepackage{multirow} %Gestion des lignes
\usepackage{outlines} %itemize avec profondeur
%A désactiver, bug lignes pas jointifs sur les tableaux
%\usepackage{tabularx} %Tableaux
%\usepackage{nicematrix}%faire des tableau avec des blocs plus facilement

%Packages Figures et graphiques
\usepackage{authblk}  %footnote et auteurs
\usepackage{graphics} %inclusion de figures
\usepackage{graphicx} %inclusion de figures
\usepackage{pgfplots} %plot
\usepackage{pgf}	  %images 
\usepackage{placeins} %eviter que les images soit trop décalés
\usepackage{pstricks,pst-node} %Graphiques
\usepackage{tikz}     %Tikz <3
\usepackage{wrapfig}  %figure avec texte à coté

\usepackage[figurename=]{caption} %réglage des captions
\captionsetup{justification=centering}
%\captionsetup[figure]{font=small,labelfont=small,labelformat=dash}
\setlength\intextsep{4pt}         %espace après wrapfigure
\setlength{\columnseprule}{1pt}   %ligne entre deux colonnes

%tikz
\pgfplotsset{compat=1.17}
